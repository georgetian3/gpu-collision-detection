\section{背景}

碰撞检测通常分为宽相位碰撞检测(broad phase collision detection)和窄相位碰撞检测(narrow phase collision detection)\cite{nvidiaCollisionDetection}。宽相位碰撞检测用于快速粗略地将完全没法碰撞的物体去除掉,然后对可能碰撞的物体进行更精确的但通常更慢的窄相位碰撞检测。由于预期仿真的物体(球,正/长方体,四面体)比较简单,此项目会使用简单的窄相位碰撞检测算法,而聚焦在宽相位碰撞检测。

\section{算法概述}

朴素的算法会将 $n$ 个物体中的每一个物体与其他 $n-1$ 个物体进行碰撞检测,其时间复杂度为 $O(n^2)$。对于几万个物体,这个算法的速度不够快。

为了降低时间复杂度,此项目使用的数据结构是层级包围体树。一个包围可以包含物体或其他包围体,必须包含它们的所有点。为了加快计算,通常会使用最小轴对齐包围盒(minimum axis-aligned bounding box),即最小的,各个边与三个坐标轴平行的长方体。
包围体可以形成树结构,约束条件是叶节点必须是物体本身,而所有其他节点都是包围体,并且任意一个内部节点必须是它的子节点的包围盒。
在便利层级包围体树时,若一个物体与某一个节点表示的包围体没有碰撞,则物体不会与此包围体的子包围体/物体碰撞,因此可以跳过子节点。对于理想构建的树而言,时间复杂度可以降低到 $O(n\log n)$。

以下会讨论算法的实现细节以及可以利用 GPU 加速的地方。

\section{树的构建}

此项目的碰撞检测对象是动态的物体,即每次物体的位置更新后需要更新层级包围体树并做一次碰撞检测。目前存高效的算法,可以一次性构建一棵树并每次更新同一棵树 \cite{Wald2008}。但是考虑在此场景中,物体的位置可能会有巨大的变化,因此选择每次进行碰撞检测时重新构建全新的树。

此项目采用了线性层级包围体算法 \cite{Lauterbach2009},将所有物体排序,并划分区间,将同一个区间的物体放在同一个包围体中。由于更小的包围体能够更精确地拟合物体,则目标是将位置类似的物体放在同一个包围体中。将三维位置信息映射到一唯空间并且保留位置的局部性,可以使用 Z 阶曲线(Z-order curve,也称 Morton order/code)\cite{wikipediaZorderCurve}。一个三维位置的 Z 值是位置的三个值的二进制表示中每个比特的交错。比如:

\begin{codeblock}
                        x = 0.00101010 -> 0.0  0  1  0  1  0  1  0
                        y = 0.10100101 -> 0. 1  0  1  0  0  1  0  1
                        z = 0.10101010 -> 0.  1  0  1  0  1  0  1  0
                             Z(x, y, z) = 0.011000111000101010101010
\end{codeblock}

计算到 Z 值后需要进行排序,而排序是一个可并行化的操作。TODO

排序后,根节点就覆盖 $[0, n-1]$ 区间内的所有物体,而它的左子树和右子树覆盖的区间分别为 $[0, i]$ 和 $[i+1,n-1]$,其中 $i$ 是某一个合适的分割点。这样递归地计算区间,直到区间内只有一个物体,它就成为叶节点。

包围体的是自底向上计算的。叶节点的包围体是物体的包围体,而内部节点的包围体是它的两个子节点的包围体的并。如果使用包围盒,可以用长方体的对角线上的两个点表示。为了简化计算,可以取 $x,y,z$ 分别最小和分别最大的两个角。在 GPU 上可以为每个叶节点创建一个线程,计算完了一个节点的包围体就计算父节点的包围体。为了避免重复地计算以及保证两个子节点的包围体也计算完毕,可以使用一个原子标志,使得第一个处理此节点的线程直接返回,而让第二个处理此节点的线程正真进行计算。

\section{树的遍历}


\bibliography{bibs}