\section{碰撞检测的加速算法设计}

\subsection{背景}

碰撞检测通常分为宽相位碰撞检测(broad phase collision detection)和窄相位碰撞检测(narrow phase collision detection)\cite{nvidiaCollisionDetection}。宽相位碰撞检测用于快速粗略地将完全没法碰撞的物体去除掉,然后对可能碰撞的物体进行更精确的但通常更慢窄相位碰撞检测。由于预期仿真的物体(球,正/长方体,四面体)比较简单,此项目会使用简单的窄相位碰撞检测算法,而聚焦在宽相位碰撞检测。

\subsection{算法概述}

朴素的算法会将 $n$ 个物体中的每一个物体与其他 $n-1$ 个物体进行碰撞检测,其时间复杂度为 $O(n^2)$。


此项目使用的数据结构是层级包围体树。一个包围体必需包含它的物体的所有点。为了加快计算,通常会使用最小轴对齐包围盒(minimum axis-aligned bounding box),即最小的,各个边与三个坐标轴平行的长方体。
包围体可以形成树结构,约束条件是叶节点必须是物体本身,而所有其他节点都是包围体,并且任意一个内部节点必须是它的子节点的包围盒。
在便利层级包围体树时,若一个物体与某一个节点表示的包围体没有碰撞,则物体不会与此包围体的子包围体/物体碰撞,因此可以跳过子节点。对于理想构建的树而言,时间复杂度可以降低到 $O(n\log n)$。

\subsection{树的构建}

此项目的碰撞检测对象是动态的物体,即需要◊每次物体的位置更新后做一次碰撞检测,因此需要更新层级包围体树。目前存在 \href{https://web.archive.org/web/20140113213158/http://visual-computing.intel-research.net/publications/papers/2008/async/AsyncBVHJournal2008.pdf}{高效的算法},可以更新重复使用同一个树,但是考虑到此在此场景中,物体的位置可能会有巨大的变化,因此选择每次进行碰撞检测时重新构建全新的树。

我们采用 \href{https://luebke.us/publications/eg09.pdf}{线性层级包围体算法},将所有物体排序,并划分区间,而在同一个区间的物体会在同一个包围体中。由于更小的包围体能够更精确地拟合物体,则目标是将位置类似的物体放在同一个包围体中。将三唯位置信息映射到一唯空间并且保留位置的局部性,可以使用 Z 阶曲线(Z-order curve,也称 Morton order/code)。一个三唯位置的 Z 值由位置的三个值的二进制表示中每个比特的交错而计算的。比如:

\begin{codeblock}
x = 0.00101010 -> 0.0  0  1  0  1  0  1  0
y = 0.10100101 -> 0. 1  0  1  0  0  1  0  1
z = 0.10101010 -> 0.  1  0  1  0  1  0  1  0  
                = 0.011000111000101010101010
\end{codeblock}

\subsection{树的遍历}

\subsubsection{其他算法}


\subsection{窄相位碰撞检测}

\section{GPU 实现的思路设计}

here \cite{nvidiaTreeTraversal} f \cite{nvidiaTreeConstruction}

\bibliography{bibs}
